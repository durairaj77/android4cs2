% Learning Data Structures and Algorithms with Android Development
% Copyright (C) 2010-2011 Goadrich, Jennings, Jadud
%
% Permission is granted to copy, distribute and/or modify this
% document under the terms of the GNU Free Documentation License,
% Version 1.1  or any later version published by the Free Software
% Foundation; with no Invariant Sections, no Front-Cover Texts,
% and no Back-Cover Texts.
% This distribution includes a file named fdl.tex that contains the text
% of the GNU Free Documentation License.  If it is missing, you can obtain
% it from www.gnu.org or by writing to the Free Software Foundation,
% Inc., 59 Temple Place - Suite 330, Boston, MA 02111-1307, USA.
%
% ``Learning Data Structures and Algorithms with Android Development'' 
% is intended to support the introduction of 
% data structures and algorithms to students in their second programming course. 
% It is based in the Java programming language, and is written around a series 
% of assignments adapted from those presented in "Data Structures and Algorithms in Java"
% by Peter Drake. 

% The LaTeX template for this book is based on that used by Downey in his book
% ``How to think like a (Python) Programmer.'' The template itself
% came with the following license:

% LaTeX source for textbook ``How to think like a (Python) Programmer''
% Copyright (c)  2007  Allen B. Downey.
% Permission is granted to copy, distribute and/or modify this
% document under the terms of the GNU Free Documentation License,
% Version 1.1  or any later version published by the Free Software
% Foundation; with no Invariant Sections, no Front-Cover Texts,
% and no Back-Cover Texts.
% This distribution includes a file named fdl.tex that contains the text
% of the GNU Free Documentation License.  If it is missing, you can obtain
% it from www.gnu.org or by writing to the Free Software Foundation,
% Inc., 59 Temple Place - Suite 330, Boston, MA 02111-1307, USA.
%

\documentclass[10pt]{book}
%\documentclass[cup6a]{cupbook}
\usepackage{pslatex}
\usepackage{url}
\usepackage{fancyhdr}
\usepackage{graphicx}
\usepackage{amsmath, amsthm, amssymb}
\usepackage{makeidx}
\usepackage{setspace}
\usepackage{hevea}
\usepackage{upquote}
\usepackage{svn-multi}
%\usepackage[draft,light]{draftcopy}

\newtheorem{ex}{Exercise}[chapter]

\newcommand{\thetitle}{Learning Data Structures and Algorithms with Android Development}

% Version information is inline on the title page.
%\newcommand{\theversion}{0.0.1)}

%%%%%%
% LaTeX Interaction
%%%%%%
\svnidlong
{$LastChangedBy$}
{$LastChangedRevision$}
{$LastChangedDate$}
{$HeadURL$}

\newcommand{\svnversioninfo}{Subversion revision \svnrev, last edited by \svnauthor \\
\svndate}


\makeindex

\begin{document}

\frontmatter


% LATEXONLY

\input{latexonly}

\begin{latexonly}

\renewcommand{\blankpage}{\thispagestyle{empty} \quad \newpage}

% TITLE PAGES FOR LATEX VERSION

%-half title--------------------------------------------------
\thispagestyle{empty}

\begin{flushright}
\vspace*{2.5in}

\begin{spacing}{3}
{\huge \thetitle}
\end{spacing}

\vspace{0.25in}

\svnversioninfo

\vfill

\end{flushright}

%--verso------------------------------------------------------

\blankpage
\blankpage
%\clearemptydoublepage
%\pagebreak
%\thispagestyle{empty}
%\vspace*{6in}

%--title page--------------------------------------------------
\pagebreak
\thispagestyle{empty}

\begin{flushright}
\vspace*{2.5in}

\begin{spacing}{3}
{\huge \thetitle}
\end{spacing}

\vspace{0.25in}

\svnversioninfo

\vspace{1in}


{\Large
Mark Goadrich\\
Jacob Jennings\\
Matthew Jadud\\
}


\vspace{0.5in}

\vfill

\end{flushright}


%--copyright--------------------------------------------------
\pagebreak
\thispagestyle{empty}

{\small
Copyright \copyright ~2010 Mark Goadrich, Jacob Jennings, Matthew Jadud.

Revision history, mostly:

\begin{description}

\item[September 2010:] Initial Version by Mark Goadrich
\item[September 2011:] ``Beetle Game'' by Matt Jadud

\end{description}

\vspace{0.2in}

Permission is granted to copy, distribute, and/or modify this document
under the terms of the GNU Free Documentation License, Version 1.1 or
any later version published by the Free Software Foundation; with no
Invariant Sections, no Front-Cover Texts, and with no Back-Cover Texts.

The GNU Free Documentation License is available from {\tt www.gnu.org}
or by writing to the Free Software Foundation, Inc., 59 Temple Place,
Suite 330, Boston, MA 02111-1307, USA.

The original form of this book is \LaTeX\ source code.  Compiling this
\LaTeX\ source has the effect of generating a device-independent
representation of a textbook, which can be converted to other formats
and printed.

The \LaTeX\ source for this book is available from
{\tt http://mark.goadrich.com/courses/csc207f09/book}

\vspace{0.2in}

} % end small

\end{latexonly}


% HTMLONLY

\begin{htmlonly}

% TITLE PAGE FOR HTML VERSION

{\Huge \thetitle}

% NOTE: MCJ 20110117
% Why is Mark so large? 
{\Large Mark Goadrich}

Version \theversion

\setcounter{chapter}{-1}

\end{htmlonly}

\chapter{Preface}

\section*{Origins of this book}

Mark Goadrich\\
Shreveport LA\\

Mark is an Assistant Professor of Computer Science
at Centenary College of Louisiana, and is the 
Broyles Eminent Scholars Chair of Computational Mathematics.

Matthew Jadud\\
Matt is an Assistant Professor Computer Science
at Allegheny College in Meadville, Pennsylvania.

\normalsize

\clearemptydoublepage

% TABLE OF CONTENTS
\begin{latexonly}

\tableofcontents

\clearemptydoublepage

\end{latexonly}

% START THE BOOK
\mainmatter

\part{Object-Oriented Programming}

\chapter{Rolling a Die}

\chapter{Playing with Beetles}

Woot. Beetles.

\chapter{Domineering}

\chapter{Flipping a Domino}

\chapter{Blinking Lights}

\part{Stacks, Queues and Lists}

\chapter{Idiot's Delignt}

\chapter{War}

\chapter{Go Fish}

\part{Trees}

\chapter{Questions}

\chapter{Ghost}

\part{Appendix}

\chapter{Opening Projects in Eclipse}
\section{Setup}
First, download Eclipse

Install Android SDK

Download all versions and tools

Set up so Device can be found on computer

Install Eclipse ADT Plugin

Set location of Android in Eclipse

\section{Project}

Download the tar.gz file

Import

General

Existing Projects into Workspace

Select archive file

Finish

Write missing classes, or copy from DSAJ text.

Run As : Android Application.

\section{Problems}

\subsection{Version of Java}
Android requires .class compatibility set to 5.0. Please fix project properties.

Android Tools -> Fix Project Properties

\subsection{Missing R.java}

ERROR: Unable to open class file \/...\/workspace\/DieRoller\/gen\/com\/googlecode\/android4cs2\/dieroller\/R.java: No such file or directory

Ok, just Run the project and it will be created.

\subsection{Missing /gen directory}

Clean the Project. Project -> Clean

\subsection{Fatal Java Error}

A fatal error has been detected by the Java Runtime Environment.

This is because you are trying to run the project as a Java application,
not an Android application. Caused by having main method in other files, 
running them instead of main Activity file. 
Wipe and reset your Launch properties,
make sure you are choosing Android Project.



\end{document}
